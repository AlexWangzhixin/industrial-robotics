\documentclass[a4paper,12pt]{article}
\usepackage[UTF8]{ctex} % 支持中文
\usepackage{amsmath, amssymb, amsfonts}
\usepackage{graphicx}
\usepackage{geometry}
\usepackage{listings}
\usepackage{xcolor}
\usepackage{float}
\usepackage{hyperref}
\usepackage{booktabs}

\geometry{left=2.5cm, right=2.5cm, top=2.5cm, bottom=2.5cm}

% 代码高亮设置
\definecolor{codegreen}{rgb}{0,0.6,0}
\definecolor{codegray}{rgb}{0.5,0.5,0.5}
\definecolor{codepurple}{rgb}{0.58,0,0.82}
\definecolor{backcolour}{rgb}{0.95,0.95,0.92}

\lstdefinestyle{mystyle}{
	backgroundcolor=\color{backcolour},   
	commentstyle=\color{codegreen},
	keywordstyle=\color{magenta},
	numberstyle=\tiny\color{codegray},
	stringstyle=\color{codepurple},
	basicstyle=\ttfamily\footnotesize,
	breakatwhitespace=false,         
	breaklines=true,                 
	captionpos=b,                    
	keepspaces=true,                 
	numbers=left,                    
	numbersep=5pt,                  
	showspaces=false,                
	showstringspaces=false,
	showtabs=false,                  
	tabsize=2
}
\lstset{style=mystyle}

\title{\textbf{工业机器人作业报告 (Assignment 2)}}
\author{学号: [Your Student ID] \\ 姓名: [Your Name]}
\date{\today}

\begin{document}
	
	\maketitle
	\tableofcontents
	\newpage
	
	\section{练习 1:基础理论 (Exercise 1)}
	
	\subsection{1. 正向运动学与逆向运动学}
	
	\textbf{正向运动学 (Forward Kinematics, FK)}
	正向运动学是研究如何根据机器人关节变量(如关节角度 $\vec{\theta}$)计算末端执行器在空间中的位置和姿态的学科。
	\begin{equation}
		T_{sb}(\vec{\theta}) = f(\vec{\theta})
	\end{equation}
	其中,$T_{sb} \in SE(3)$ 是末端相对于基坐标系的齐次变换矩阵。
	\textit{应用示例}:机器人路径规划中的碰撞检测、可视化仿真、末端位置监控。
	
	\textbf{逆向运动学 (Inverse Kinematics, IK)}
	逆向运动学是研究如何根据末端执行器的期望位置和姿态 ($T_{sd}$),反求出机器人所需关节变量 $\vec{\theta}$ 的学科。
	\begin{equation}
		\vec{\theta} = f^{-1}(T_{sd})
	\end{equation}
	\textit{应用示例}:焊接路径规划、拾取与放置 (Pick and Place) 任务。
	
	\subsection{2. 逆运动学的数值求解方法}
	
	对于一般的空间机械臂,特别是冗余机械臂,解析解往往难以获得,常采用\textbf{牛顿-拉夫森 (Newton-Raphson)} 迭代法进行数值求解。其核心步骤如下:
	
	\begin{enumerate}
		\item \textbf{初始化}:设定初始猜测关节角 $\vec{\theta}_0$。
		\item \textbf{正运动学计算}:计算当前关节角下的末端姿态 $T_{sb}(\vec{\theta}_k)$。
		\item \textbf{误差计算}:计算当前姿态与期望姿态 $T_{sd}$ 之间的误差旋量 $\mathcal{V}_b$ (在物体坐标系下) 或 $\mathcal{V}_s$ (在空间坐标系下)。
		\begin{equation}
			[\mathcal{V}_b] = \log(T_{sb}^{-1} T_{sd}) \quad \text{或} \quad [\mathcal{V}_s] = \log(T_{sd} T_{sb}^{-1})
		\end{equation}
		\item \textbf{雅可比计算}:计算对应的物体雅可比 $J_b(\vec{\theta}_k)$ 或空间雅可比 $J_s(\vec{\theta}_k)$。
		\item \textbf{更新关节角}:
		\begin{equation}
			\vec{\theta}_{k+1} = \vec{\theta}_k + J^{\dagger}(\vec{\theta}_k) \mathcal{V}_{err}
		\end{equation}
		其中 $J^{\dagger}$ 为雅可比矩阵的伪逆 (Pseudo-inverse),用于处理冗余自由度或奇异性。
		\item \textbf{收敛判断}:若 $||\mathcal{V}_{err}|| < \epsilon$,则停止;否则回到步骤2。
	\end{enumerate}
	
	\section{练习 2:平面机械臂分析 (Exercise 2)}
	
	基于图示的3R平面结构(实际上根据题目描述主要关注刚体运动),已知 $L=1.0$m, $d=0.4$m, $\theta=30^\circ$, $\dot{\theta}=1$ rad/s。
	
	\subsection{2.1 点 P 的位置与速度}
	点 P 在 {b} 系中为原点。在固定坐标系 {s} 下:
	\begin{equation}
		\vec{p}_P = \begin{bmatrix} L + d\sin\theta \\ L - d\cos\theta \\ 0 \end{bmatrix} = \begin{bmatrix} 1 + 0.4\sin(30^\circ) \\ 1 - 0.4\cos(30^\circ) \\ 0 \end{bmatrix} = \begin{bmatrix} 1.2 \\ 0.6536 \\ 0 \end{bmatrix} \text{m}
	\end{equation}
	注意:此处坐标系定义依据题目图示(P点相对于旋转中心的偏移)。
	速度 $\dot{\vec{p}}_P$ 为:
	\begin{equation}
		\dot{\vec{p}}_P = \vec{\omega} \times \vec{r} = \begin{bmatrix} 0 \\ 0 \\ \dot{\theta} \end{bmatrix} \times \begin{bmatrix} d\sin\theta \\ -d\cos\theta \\ 0 \end{bmatrix} = \begin{bmatrix} d\cos\theta \cdot \dot{\theta} \\ d\sin\theta \cdot \dot{\theta} \\ 0 \end{bmatrix}
	\end{equation}
	
	\subsection{2.2 齐次变换矩阵 $T_{sb}$}
	\begin{equation}
		T_{sb} = \begin{bmatrix} 
			\cos\theta & -\sin\theta & 0 & p_{Px} \\
			\sin\theta & \cos\theta & 0 & p_{Py} \\
			0 & 0 & 1 & 0 \\
			0 & 0 & 0 & 1
		\end{bmatrix}
	\end{equation}
	
	\subsection{2.3 旋量 (Twist) $\mathcal{V}_b$ 与 $\mathcal{V}_s$}
	\textbf{空间旋量 $\mathcal{V}_s$}:
	由于旋转中心在 $(L, L, 0)$,则 $\mathcal{V}_s = [\omega_z, v_x, v_y, v_z]^T$。
	$\omega = 1$ rad/s。$v = -\omega \times q$,其中 $q=[L, L, 0]^T$。
	\begin{equation}
		\mathcal{V}_s = [0, 0, 1, L, -L, 0]^T
	\end{equation}
	
	\textbf{物体旋量 $\mathcal{V}_b$}:
	\begin{equation}
		\mathcal{V}_b = \text{Ad}(T_{sb}^{-1}) \mathcal{V}_s
	\end{equation}
	通过 Matlab 验证计算结果。
	
	\section{练习 3:Unitree 机器狗腿部运动学 (Exercise 3)}
	
	\subsection{3.1 轨迹生成}
	任务要求生成用于翻越障碍的椭圆轨迹。给定起点 $\mathbf{x}_1=(0,0)$,终点 $\mathbf{x}_2=(0.25, 0.2)$,离心率 $e=0.9$。
	计算得长半轴 $a \approx 0.1601$ m,短半轴 $b \approx 0.0698$ m,旋转角 $w \approx 0.6747$ rad。
	轨迹方程参数化为 $\beta \in [\pi, 0]$。
	
	\subsection{3.2 解析逆运动学 (Analytical IK)}
	我们将问题转化为 2R 平面机械臂的逆解。需要将足端坐标系 $\{F\}$ 下的坐标转换到关节 1 坐标系 $\{J1\}$ 下。
	\begin{itemize}
		\item 变换关系:$\mathbf{p}_{J1} = \mathbf{p}_F - (0, 0.5)$
		\item 求解方法:利用余弦定理求解膝关节角 $\theta_2$,利用几何法求解髋关节角 $\theta_1$。
		\item 解的选择:根据“狗腿”构型,选择膝盖向后的解(elbow-back/knee-back)。
	\end{itemize}
	
	\subsection{3.3 \& 3.4 数值逆运动学与速度约束}
	我们利用雅可比矩阵 $J(\theta)$ 进行速度控制。
	\begin{equation}
		\dot{\theta} = J^{\dagger}(\theta) \mathcal{V}_{foot}
	\end{equation}
	通过设定关节速度上限 $\dot{\theta}_{max} = 5.0$ rad/s,我们通过 Matlab 脚本迭代搜索了最大允许的轨迹速度参数 $\dot{\beta}$。仿真结果表明,当 $\dot{\beta} \approx 1.0$ rad/s 时,关节速度保持在约束范围内。
	
	\section{练习 4:7自由度冗余机械臂 (Exercise 4)}
	
	\subsection{4.1 线性运动}
	针对7自由度 (S-R-S配置) 机械臂,我们实现了基于数值逆运动学的线性路径跟踪。
	由于 $n=7 > 6$,雅可比矩阵非方阵。我们使用伪逆 $J^{\dagger} = J^T(JJ^T)^{-1}$ 来求解 $\dot{\theta}$,该方法隐式地最小化了关节速度范数 $||\dot{\theta}||$。
	
	\subsection{4.2 \& 4.3 三角形速度规划 (Triangular Time-Scaling)}
	推导了三角形速度曲线 $s(t)$。
	\begin{itemize}
		\item 最大速度 $v_{max} = 2/T_f$。
		\item 加速段 ($0 \le t \le T_f/2$): $s(t) = \frac{2}{T_f^2}t^2$。
		\item 减速段 ($T_f/2 < t \le T_f$): 对称减速。
	\end{itemize}
	Matlab 代码实现了该规划并驱动机械臂完成了点到点 (P2P) 运动。
	
	\subsection{4.4 梯形速度规划 (Trapezoidal Time-Scaling)}
	在三角形规划基础上引入巡航阶段(匀速段),由上升时间 $t^*$ 定义。
	\begin{equation}
		v_{max} = \frac{1}{T_f - t^*}
	\end{equation}
	这允许更平滑的加减速过程,减少了对电机的冲击。
	
	\section{练习 5:6自由度拟人机械臂 (Exercise 5)}
	
	\subsection{5.1 运动学建模}
	该机械臂具有拟人构型(腰-肩-肘-球形腕)。使用指数积公式 (PoE) 建立正运动学模型。
	\begin{equation}
		T(\theta) = e^{[\mathcal{S}_1]\theta_1} \dots e^{[\mathcal{S}_6]\theta_6} M
	\end{equation}
	通过蒙特卡洛法绘制了受关节限位约束的工作空间点云。
	
	\subsection{5.2 \& 5.3 雅可比与逆运动学}
	编写了自定义函数计算空间雅可比矩阵 $J_s(\theta)$,并利用其实现了 Newton-Raphson 迭代法求解逆运动学。
	针对目标位姿 $T_{se}$ 和 $\vec{x}_e$,求解器均能在 10 次迭代内收敛到精度 $10^{-4}$。
	
	\subsection{5.4 \& 5.5 轨迹规划}
	\begin{itemize}
		\item \textbf{P2P 轨迹}:使用三次多项式插值 (Cubic Scaling) 生成关节空间轨迹。分析发现 $T_f=0.5$s 时关节速度超过 30 deg/s 的限制,需延长至约 3.0s 以上。
		\item \textbf{多路点轨迹}:实现了 Home $\to$ Target1 $\to$ Target2 $\to$ Home 的连续轨迹规划,总时长 25s,保证了运动的平滑性和速度约束的满足。
	\end{itemize}
	
	\section{总结}
	本作业涵盖了从平面简单机构到复杂冗余及拟人机械臂的完整建模与控制流程。通过自主开发 Matlab 求解器(不依赖工具箱的高级函数),深入理解了旋量理论、雅可比矩阵及数值优化在机器人学中的核心作用。
	
\end{document}